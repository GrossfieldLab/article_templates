\documentclass[11pt]{article}

\usepackage[numbers,super,sort&compress,comma]{natbib}
\usepackage[pdftex,colorlinks=true,urlcolor=black,filecolor=black,linkcolor=black,citecolor=black,breaklinks=true]{hyperref}
\usepackage{hypernat}
\usepackage[pdftex]{graphicx}
\usepackage{palatino} % make fonts Helvetica for sans, palatino for serif
\usepackage[usenames,dvipsnames]{color}


\newcommand{\response}[1]{\textcolor{Mahogany}{\textbf{#1}}}

\begin{document}

I thank the reviewers for their kind and constructive comments. I hope they agree that incorporating their suggestions has significantly improved the manuscript. I've listed the comments requiring responses below, with my response colored \response{in mahogony}.

\section{Reviewer A}

1) My biggest concern is simply that the people who ought to read it,
won't do so (even if directed) because it is longer than what users
typical want to read when they are in search of a fix to their
problem. In this the article shares the same shortcoming with the
well-known online essays [1] and [2].

I'd like a condensed summary of about one paragraph (maybe as a
box) that I can just paste in an email or put on a website,
together with a link to the full version.

\response{I have added references to the Raymond and Tatham essays -- I'm
somewhat embarrassed I didn't think to add them myself, since they clearly
influenced my thinking. I would argue that the checklists make
a good short punchy summary, as the reviewer suggested. However, I agree having
a plain-text version wouldn't hurt, so I've added a ``Summary'' paragraph before
the checklists.}

2) Does the author make a distinction between software that is
advertised as useable (e.g. through project websites, presence of
forums, issue tracker, ...) and software that has been made
available (archive, repository, supplementary material)?

Can and should a user assume that a developer of the latter will
provide any support?

Perhaps this could be made clearer.

My personal view is that FAIR principles of research make it
essential that code is published together with the research (code
dump with a license). However, for code dumps I wouldn't require
authors to respond to more than the most fundamental inquiries
(e.g., to respond when bugs are found that question the integrity
of the published research). If more is demanded of authors then
that disincentivizes publishing the code with the research.

On the other hand, if code is advertised (presumably with the
expectation of citations or some other benefit to the developer)
then I agree with the author that developers owe their users some
of their time.

\response{My focus was mostly on code that is intended to be used. I've added a
paragraph at the end of the section for developers laying out my thoughts on
this distinction.}

3) "Please open a PR" (or merge request) is a phrase that can strike
fear in the heart of a novice. It might be helpful to demystify
this important part of collaborative software development. Maybe
the a PLoS Comp Biol article [5] or something similar that touches
upon basic software engineering that could provide more guidance.

\response{I agree, and I'd argue that the relevance here is that it falls to the
developers to provide a less intimidating way to contribute. I'd rather not talk
about specific software engineering technologies here, though, just because I
feel like it's a topic I couldn't do justice to without doubling the length of
the paper.}



4) It might be useful to prepare the new community member that
conversations — although respectfule — are often to the point and
that most developers won't sugar-coat their words: if something is
wrong, it will be said just like that because ultimately developers
are concerned with maintaining the health of the code base and the
project and have to be clear about what is acceptable.

New users should not feel offended by the phrase "Feel free to work
on it, submit a PR, and we can discuss there." or similar. Unless
you pay the developers to implement what you want, they will likely
encourage you to do most of the work yourself and then spend some
of their scarce time guiding you to a solution.

\response{This is a good point (although I'd argue that developers ought to be a
bit more courteous than that). I added a couple of sentences at the end of the
bug report section, appended to the paragraph about feature requests.}


5) Pointing out how much one can learn by interacting with experienced
developers is important. This is a chance to learn from the best
people in the field.

Furthermore, the networks that you build as part of a software
project might last you a life time.

\response{I added a sentence mentioning this to the section on contributing to the community.}


6) I appreciate that the article primarily originates in the
experience of the author. I would nevertheless find it useful if,
where possible, other sources could also be cited as validation and
to show that indeed these are broadly shared values.

I found the following references useful but I am emphatically not
implying that they must be included in the article.

Ref [3] has a good section on the importance of communities.

\response{I'm embarrassed to admit I hadn't read either of the suggested papers.
I've now read them and added references to both.}


\section{Reviewer B}

1) This point is already implicitly transparent throughout the paper, but it
might be useful to state this more explicitly right in Section 3.1 as it
(partially) motivates everything that follows: The easier you make it for the
developer to narrow down/reproduce the issue the higher the chance they will
help you. I think it's worth pointing out that developers are usually handling
multiple issues at the same time and typically tend to assign lower priority to
issues that are harder and time-consuming.

\response{This is a very good point. I added a new paragraph under the description of a bug report.}

2) In section 3.5 it is stated: "Once you’ve become reasonably competent with a
piece of software, it’s worth looking for ways to contribute back. [...] You can
work on the documentation". This is a wonderful suggestion. However, I'm afraid
that stated this way, users that don't feel competent enough will be
hesitant/justified in contributing. I believe proposing small changes to the
documentation often does not require competency. On the contrary, devs and
experienced users typically take for granted a lot of things that might be
missing from the docs. My point is that it might be a good idea to rephrase a
bit to encourage even first users to propose changes to the docs as the best way
to give back some of the time the devs/community has dedicated to them.

\response{This is a very good point. The new version begins ``Once you've begun using a piece of software, it's worth looking for ways to
contribute back.  This can take many forms, not all of which involve writing
code (though of course contributing bug fixes or new features is extremely
valuable). You don't have to be particularly expert -- sometimes the best
documentation comes from the inexperienced or newly experienced users, who
vividly remember not understanding something in the manual and can suggest
revisions that would make it clearer. ''}

\section{Reviewer C}

\response{Thank you for your kind words!}

\end{document}
