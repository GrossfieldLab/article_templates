%%%%%%%%%%%%%%%%%%%%%%%%%%%%%%%%%%%%%%%%%%%%%%%%%%%%%%%%%%%%
%%% LIVECOMS ARTICLE TEMPLATE FOR TRAINING ARTICLE
%%% ADAPTED FROM ELIFE ARTICLE TEMPLATE (8/10/2017)
%%%%%%%%%%%%%%%%%%%%%%%%%%%%%%%%%%%%%%%%%%%%%%%%%%%%%%%%%%%%
%%% PREAMBLE
\documentclass[9pt,training]{livecoms}
% Use the 'onehalfspacing' option for 1.5 line spacing
% Use the 'doublespacing' option for 2.0 line spacing
% Use the 'lineno' option for adding line numbers.
% Use the "ASAPversion' option following article acceptance to add the DOI and relevant dates to the document footer.
% Use the 'pubversion' option for adding the citation and publication information to the document footer, when the LiveCoMS issue is finalized.
% The 'training' option for indicates that this is a training article.
% Omit the bestpractices option to remove the marking as a LiveCoMS paper.
% Please note that these options may affect formatting.

\usepackage{lipsum} % Required to insert dummy text
\usepackage[version=4]{mhchem}
\usepackage{siunitx}
\DeclareSIUnit\Molar{M}
\usepackage[italic]{mathastext}
\graphicspath{{figures/}}

%%%%%%%%%%%%%%%%%%%%%%%%%%%%%%%%%%%%%%%%%%%%%%%%%%%%%%%%%%%%
%%% IMPORTANT USER CONFIGURATION
%%%%%%%%%%%%%%%%%%%%%%%%%%%%%%%%%%%%%%%%%%%%%%%%%%%%%%%%%%%%

\newcommand{\versionnumber}{0.1}  % you should update the minor version number in preprints and major version number of submissions.
\newcommand{\githubrepository}{\url{https://github.com/GrossfieldLab/software-community}}

%%%%%%%%%%%%%%%%%%%%%%%%%%%%%%%%%%%%%%%%%%%%%%%%%%%%%%%%%%%%
%%% ARTICLE SETUP
%%%%%%%%%%%%%%%%%%%%%%%%%%%%%%%%%%%%%%%%%%%%%%%%%%%%%%%%%%%%
\title{How to be a good member of a scientific software community [Article v\versionnumber]}

\author[1*]{Alan Grossfield}
%\author[1,2\authfn{1}\authfn{3}]{Firstname Middlename Familyname}
%\author[2\authfn{1}\authfn{4}]{Firstname Initials Surname}
%\author[2*]{Firstname Surname}
\affil[1]{University of Rochester Medical Center, Department of Biochemistry and Biophysics}
%\affil[2]{Institution 2}

\corr{alan_grossfield@urmc.rochester.edu}{AG}  % Correspondence emails.  FMS and FS are the appropriate authors initials.
%\corr{email2@example.com}{FS}

\orcid{Alan Grossfield}{0000-0002-5877-2789}
%\orcid{Author 2 name}{EEEE-FFFF-GGGG-HHHH}

%\contrib[\authfn{1}]{These authors contributed equally to this work}
%\contrib[\authfn{2}]{These authors also contributed equally to this work}

%\presentadd[\authfn{3}]{Department, Institute, Country}
%\presentadd[\authfn{4}]{Department, Institute, Country}

\blurb{This LiveCoMS document is maintained online on GitHub at \githubrepository; to provide feedback, suggestions, or help improve it, please visit the GitHub repository and participate via the issue tracker.}

%%%%%%%%%%%%%%%%%%%%%%%%%%%%%%%%%%%%%%%%%%%%%%%%%%%%%%%%%%%%
%%% PUBLICATION INFORMATION
%%% Fill out these parameters when available
%%% These are used when the "pubversion" option is invoked
%%%%%%%%%%%%%%%%%%%%%%%%%%%%%%%%%%%%%%%%%%%%%%%%%%%%%%%%%%%%
\pubDOI{10.XXXX/YYYYYYY}
\pubvolume{<volume>}
\pubissue{<issue>}
\pubyear{<year>}
\articlenum{<number>}
\datereceived{Day Month Year}
\dateaccepted{Day Month Year}

%%%%%%%%%%%%%%%%%%%%%%%%%%%%%%%%%%%%%%%%%%%%%%%%%%%%%%%%%%%%
%%% ARTICLE START
%%%%%%%%%%%%%%%%%%%%%%%%%%%%%%%%%%%%%%%%%%%%%%%%%%%%%%%%%%%%

\begin{document}

\begin{frontmatter}
\maketitle

\begin{abstract}

Software is ubiquitous in modern science --- almost any project, in almost any
discipline, requires some code to work. However, many (or even most) scientists
are not programmers, and must rely on programs written and maintained by others.
A crucial but often neglected part of a scientist's training is
learning how to use new tools, and how to exist as part of a community of users.
This article will discuss key behaviors that can make the experience quicker,
more efficient, and more pleasant for the user and developer alike.


\end{abstract}

\end{frontmatter}




\section{Introduction}

Most practicing scientists (even the ones who write code as part of their
research) spend more time as consumers as opposed to producers of software.
Moreover, we are constantly learning new skills and solving new problems, which
often means learning to use new programs or new aspects of large packages. This,
combined with the complex nature of scientific software (and often the missing
or inaccurate documentation) means we have to ask for help.  Sometimes it's
because we don't know how to accomplish a specific task. Others, it's because
there's a feature we need that isn't implemented. Inevitably, there are bugs.

In all of these cases, it is necessary to interact with the people who develop
and support the code. How you go about it has an enormous impact on your
likelihood of success and how you are viewed. Asking complete strangers for help
is often intimidating, especially the first time you do it. However, the best
way to do so is rarely taught explicitly --- at best, it's something one picks
up by example, from fellow lab members or your PI.  \emph{The goal of this
paper is to reveal the hidden curriculum -- what's expected of a software
consumer, how to ask for help, how to contribute productively to a software
community (regardless of whether you can write code).}

This article is written from a the perspective of a developer of academic
open source scientific software, based on my personal experience as a developer
and user. I released my first piece of open source code, wham \cite{WHAM}, a
tool for analyzing umbrella sampling simulation,  during the summer of 2000.  My
group released LOOS\cite{Grossfield-2009, LOOS-JCC}, a suite of tools for
developing new tools to analyze molecular dynamics simulations, in 2008. Over
the years, I've interacted with hundreds of people who've tried to use one
package or the other. Most of these interactions have been positive, some
overwhelmingly so. Others were not. This article is an attempt to distill my
experiences into a concise set of recommendations.

Much of what I suggest is universal, but some points --- like the
reminder that the people providing the support are likely volunteers --- are
not.  Regardless, I view the interaction between developers and users as an
informal social contract, where each has obligations and expectations for the
other. Obviously, every software community is unique, and many have specific
conventions for interaction, but we hope that the recommendations we make are at
the very least a good starting point for new users of scientific software.



\section{Target Audience}

This paper is primarily aimed at junior scientists who are new to performing
computational research and inexperienced at asking for help. However, we hope it
will be valuable to anyone who uses software written by others to do their
research.

\section{Good practices in asking for help}

\subsection{Try to solve your own problem}
\label{ss:yourself}

There's absolutely nothing wrong with needing help to figure out how to do
something with a new piece of software. Perhaps your use case wasn't considered
in the manual, or you're seeing behavior you don't expect. Perhaps you can't
even get it to install. Asking for help is very reasonable --- the developers
want people to use their software, and with that comes an implied obligation to
support those users. However, keep in mind that the developers of scientific
software are, for the most part, volunteering their time to provide that
support, and as such it's important to respect their time.

For this reason, asking the developers for help shouldn't be the \emph{first}
thing you do. Especially for software with a large community, chances are
someone else has run into similar difficulties, and you can probably save
yourself and the developers a lot of time if you look for solutions on your own
first. Indeed, quickly debugging your use of other people's software is one of
the major skills one develops when learning to do computational research.

The first step is to read the error message carefully, if there is one; this
obviously doesn't apply in all cases, but it's often applicable and far too
often skipped. It's true that many software packages have cryptic or unhelpful
error messages, but far too often people write for help when the solution is
right in front of them. If the error message says ``Could not open file
foo.dat'', it's worth checking whether there is a file called ``foo.dat'' and if
not figuring out what was supposed to create it.

The second step is to check the manual. This too might seem trivially obvious,
but you'd be astonished how many emails developers get that are directly
answered in the manual. If the manual is short, you can read the whole thing. If
it's longer, search for relevant-seeming keywords, scan the sections that seem
to pertain to your error message. If there's an FAQ (frequently asked questions)
list, check that first; asking an FAQ, especially in a public forum, is likely
to lead to grumbling.

Next, search the internet for the error message; this works very well for
software with thousands of users, where it might very well deserve to be the
first option. It is less effective with less common packages, but it's an easy
thing to check, so you'd be foolish to skip it.  Search for the whole text of
the error (excluding any verbiage that looks like it depends on the compiler,
etc), with and without the package name. The latter can be important in the case
where the problem isn't with the package per se, but with something else in your
setup.

One valuable strategy is to construct a minimal set of circumstances that causes
the problem. Even if you initially encountered the problem while analyzing a 2
TB data set, that's not going to be very helpful for diagnosing the problem (and
you won't easily be able to share the data set with the developers so they can
diagnose it). Can you reproduce the problem using only a small portion of the
data? Is the problem present with some kinds of data (e.g. some frames of the
trajectory) but not others, and if so what is different about those frames? Can
you make the problem appear and disappear with small changes in the settings?
Testing these things out may lead you to a solution, but even if they don't
they'll help you describe the problem more precisely to the developers.

Along the same lines, it is often valuable to construct a fake or minimal data
set where the correct answer is known (or easily computed outside the program).
This is particularly important when your concern is that the program is
producing incorrect answers due to a coding error or assumptions incompatible
with the input data. The more you precisely you can narrow down what's going
wrong, the easier the problem will be to diagnose.

Finally, you can consider diving into the code. This decision depends very much
on the complexity of the package, your skill level, and how urgently you need
the solution. More often than not, comprehending the whole software package will
be too time consuming, but there are tricks and shortcuts one can use to at
least figure out where things might be going wrong. Figuring out where the error
is occurring can sometimes help you figure out why it happened, so examine the
stack trace if there is one. Alternatively, you can search the code base for the
error message, to see what drove the error; it's not uncommon for the comments
in the code to be clearer and more indicative of what could go wrong than the
error message itself. To be clear: as an end user, \emph{you are not obligated
to check the code}, but it can sometimes be a good shortcut to help you either
solve your problem on your own or give the developers what they need to solve
it. Moreover, if you do solve the problem, it's a good idea to share your
solution with the developers; most will be grateful to you for your help, and
they have comments or improvements suggested improvements. If the package is
hosted on GitHub, consider opening a pull request so your solution can be merged
into the main branch.

\subsection{Ask for help in the right place}

Once you've determined you need help, the next obvious step is to ask for it.
How should you go about doing so? Email the developer? Post in a forum? Raise an
issue on GitHub? Asking the question in the right venue will greatly increase
your chances of getting a timely answer. Step 1 is to check the documentation:
look at the manual, check the README on the GitHub repository, check the
software website, etc --- most of the time, at least one of these sources will
say how they prefer to receive bug reports, feature requests, etc. Actually,
there's a step 0: make sure you're looking at the right piece of software. This
might seem obvious, but I've received multiple requests for help with software I
had nothing to do with; once it took 8 emails back and forth before I figured
this out (but more on that problem later).

Asking in the right place is particularly important if the software is complex
or has a large user community --- there may be separate forums for bug reports,
usage help, and feature request, and asking in the wrong place will make it
unlikely the right people will see your question. Moreover, the first impression
you'll make is that you're someone who can't be bothered to read the
instructions, which will make them less inclined to help you.

\subsection{Write a good bug report}

The next step in the process is to actually communicate your issue with the
developers, generally in the form of a bug report.  The challenges in writing an
effective bug report have been discussed before in the context of free software
\cite{Raymond, Tatham}.  Doing so requires striving for two somewhat
contradictory goals: you must give the developer all of the relevant information
about the problem, as concisely as possible.

Taking the first part first: this is hard, because as a user you're usually not
an expert, \emph{so you don't know what's relevant}. The only way to know for
sure what is and isn't relevant is to develop a mental model for the problem,
which requires effort. Still, as pointed out above, your prior efforts to solve
the problem yourself (see \ref{ss:yourself}) will help you here.  Moreover, you
may find that the act of writing the note asking for help may in itself let you
solve your problem. The process is analogous to rubber duck debugging
\cite{Thomas-1999}, in that the act of figuring out how to describe a problem
clearly can lead you to a solution.

Moreover, many developers will help you with the process. For example,
GitHub-based projects have the option to set up issue templates
\cite{github-templates}; as the name implies, these templates contain a series
of questions one should answer when submitting a new issue. Although they are
generally customized for each project, these questions are also a good baseline
for the kinds of information one should supply when asking for support
elsewhere. For example, for a bug report one should

\begin{itemize}

    \item Describe the bug, including the expected behavior and how the current
    behavior differs.

    \item Give steps necessary to reproduce the bug. Include any input data
    required.

    \item Give the version of the code you're using. If you've changed anything,
    say what you've changed. If there are compile-time options, say what you
    chose.

    \item Describe the platform you're running on (e.g. the operating system for
    a conventional computer, hardware if it's a tablet, etc.). If the problem is
    related to building or installing the software, describe the build system
    (e.g. compiler version) or package manager (e.g. Conda\cite{CONDA}, Homebrew \cite{HOMEBREW}).

    \item For a command line tool, provide the command line and configuration
    information in plain text, \emph{NOT} as a screen shot. A screen shot is
    often harder to read, and at a bare minimum forces the developer to retype
    your command instead of copying and pasting, which makes it harder for them
    to spot possible typos, etc.

\end{itemize}

Following this format (or something similar) makes it much easier for the
developer to diagnose your problem, and the easier you make it the more likely
they are to solve your problem quickly. While it's not ideal behavior, busy
developers (particularly those for whom software support is a side activity) may
shy away from starting work on a problem that looks hard or uninviting, while a
clear concise description of the problem is more likely to spur immediate
action.

Sometimes, the developer will be able to answer your question based on the first
email. However, their reply will often contain questions or suggestions for
things to test. \emph{It is crucial that you read this message carefully, and
answer all of the questions to the best of your ability when you reply.} As a
rule, these questions aren't asked idly; the person providing support is trying
to help you! Far too often, users don't read the reply email carefully, don't
answer the questions, try only the first suggestion, or refuse to indicate what
it is they were actually doing. If you do this, the best case is that it will
take longer for you to get to a solution. More likely, you will frustrate the
person trying to help you, and they may become reluctant to work on your issue.
In my experience, few things are more frustrating than trying to help someone
who won't help me help them.

Feature requests are similar: you need to describe what you hope to accomplish
and how you'd like to see it work, as well as any current workaround you've
developed. Keep in mind that many features sound easy, but there can be deep
design issues or technical nuances that make their implementation difficult.
Moreover, it is possible that the developer won't be enthusiastic about your
idea, even if it is doable --- for example, it might not coincide with their
vision for the software. Or, as discussed above, academic developers are
generally not compensated for their work (obtaining NIH funds to support
software development is extremely difficult) and simply might not have the time
to work on new features. For these reasons, do not be surprised or insulted if
the developers suggest you work on a particular solution yourself and submit a
pull request. This doesn't mean they dislike your idea, just that they're busy -- they'll help guide you as best they can, but they can't do the work themselves.

\subsection{Treat people with respect}

This should go without saying, but if you're asking for help it behooves you to
do so respectfully. We're not saying that you need to be obsequious. Rather,
approach every interaction with respect: respect for people's time, respect for
their ideas, respect for their identity (including among other things race,
ethnicity, country, native language, sex or gender, or career stage). If you're
asking for help, do so with the perspective that the developer is volunteering
their time to help you, with minimal payoff to them.  If you're reporting a bug,
you're almost certainly frustrated; try not to take that frustration out on the
developers. If you're offering a new feature, keep in mind that it may not align
with the developers' vision for where the software is going.

If you're interacting with fellow users in a support forum, don't mock
questioners who are less experienced than you --- a question may seem foolish to
you, but even an uninformed or ill-directed question is an opportunity for
teaching. That said, we understand that answering the same questions repeatedly
is frustrating and exhausting; one of our goals for this document is that it
become a resource to educate new users of expected norms without consuming
developers' time and mental bandwidth.

\subsection{Contribute to the community}

Once you've begun using a piece of software, it's worth looking for ways to
contribute back.  This can take many forms, not all of which involve writing
code (though of course contributing bug fixes or new features is extremely
valuable). You don't have to be particularly expert -- sometimes the best
documentation comes from the inexperienced or newly experienced users, who
vividly remember not understanding something in the manual and can suggest
revisions that would make it clearer.  Many packages hosted on GitHub have the
documentation in the same repository, so the same pull request mechanism can be
used, but even if the docs are separate (or if you find the idea of a pull
request intimidating) you could write a new paragraph or two and send them to
the developer. If you developed a toy example as you were figuring out how to
use the software, consider sharing it. If one of the provided examples had to be
tweaked to work, share the tweaks. If the software has an online forum (e.g.
mailing list, GitHub discussions or issues, etc), you might be able to
contribute by answering questions from less experienced users.

Unlike the other recommendations, this one isn't mandatory --- it's a ``nice to
have'' rather than a ``must have''. However, there are a number of advantages to
doing it. First and foremost, you'll be making a package that's valuable to you
stronger, which will help you with your own scientific projects, while getting
feedback from leading experts in the field. However, participating productively
in a software community is also a good way to build your scientific
communication skills. Finally, helping others is an excellent way for a
relatively junior scientist to build their reputation in the community, which in
turn can lead to future collaborations or even job opportunities. At the very
least, it's always valuable when a lot of people in your field have positive
associations with your name.

\section{Obligations of software developers}

This paper has focused on what software consumers should do when asking for
help.  I suppose it could even be read as a list of complaints about bad
behavior by users (though that's not how I intend it). However, each behavior
described above has a corresponding expectation for the developers. If we expect
the users to try to solve their own problem, it behooves us to write good
documentation, give examples, and (most importantly) keep them up to date. If we
want to be contacted via GitHub only, we need to make that information clear and
easy to find. If we are frustrated by bug reports that don't give us the
information we need, we should tell people what kinds of information we need
(e.g. by providing bug report templates). In order to receive proper credit, we
need to document the appropriate way to cite the software. All of this
information needs to be clearly written out and easy to find.

Doing these things is valuable from a practical perspective, as discussed
elsewhere \cite{Heister-2013, Laayouni-2016, Wallqvist-2019}; making it easy for
people to interact with us efficiently, even if it costs us some effort up
front, will make those interactions more pleasant. Even setting that aside, it's
the right way to do science --- FAIR principles for software development
\cite{Lamprecht-2020} prioritize not just releasing code, but making it useable.
That means documenting it well and supporting people who try to use it.

Users approaching us in good faith deserve to be treated fairly and courteously.
Treating everyone with courtesy and respect, regardless of their identity, is
essential if you want to build a vibrant effective community.  If the community
is just you, the sole developer, then the solution is obvious --- when you
interact with users, do so in ways that make them feel valued and respected.

As the user and developer community grows, other people will get involved in
supporting users. It's still your obligation as a developer to maintain those
standards.  Making sure that racist, sexist, anti-LGBTQ, etc language and
behavior is unwelcome and unacceptable is simply the moral thing to do.  This
doesn't just mean use of epithets in messages (though that's clearly
unacceptable). Keep an eye on how things are named, on jokes present in the code
or messages, on whose opinions are listened to and who is ignored.  Above all,
if someone tells you a particular behavior or situation is harmful,
\emph{believe them}.  Something that feels minor (or is just a joke) to you may
hit others very differently, and part of respect is understanding and accounting
for that.

It's also important to recognize that even while English has become the default
language for science, not everyone is a native speaker, which means that some
jargon, abbreviations, and idioms may be incomprehensible to a fair portion of
your community. Moreover, an awkwardly worded question often reflects lack of
familiarity with the language rather than laziness or ignorance. It's important
to keep accessibility in mind while creating documentation or interacting with
users in a forum.

Along the same lines, a developer who berates a user for asking a stupid
question, or is rude and dismissive in a technical discussion about a new
feature, is actively damaging the community, \emph{regardless of the technical
quality of their contributions.} It's not enough for you as a developer to treat
people with respect.  To the extent that there's a community around your
software, you must attempt to make sure that community is respectful; as
developer, your words and deeds carry significant weight. If you call out bad
behavior (or even marginal behavior), that will go a long way towards setting a
good tone for your community. If you let it slide, chances are others will
assume you implicitly endorse it.

If the justice perspective doesn't persuade you, consider the practical
consequences. Bad behavior on the part of developers of your project (or your
community in general) will drive users and developers away from you. If
interacting with your community is unpleasant, folks will either use alternative
codes or use your code but not cite you. The next person with an idea for an
innovative new feature will add it to a different package, or won't do it at all
(hurting the whole community). Ask yourself: can your project really afford to
drive away talented hard-working people eager to help? Is it worth it, just so a
few people don't have to engage with basic courtesy?

Sometimes, code is published that is not particularly intended for use. For
example, it is good practice to release the scripts used to analyze data in
connection with the paper publishing the data, but many of these scripts are
one-offs, and the authors have little incentive to polish them to the point of
general useability. Rather, the scripts are essentially there as documentation,
not turn-key programs for others to use. In this case, the authors have less of
an obligation to provide support, although at a minimum laying out the software
necessary to run it (e.g. versions of python and numpy, or a Makefile and
compiler version) should be required.  That said, we recognize that some bit rot
is expected with academic code; once the student who wrote the code graduates
and moves on, it's very possible that there is no one else to maintain it.
Solving the long-term support and maintenance problem for academic software is a
major problem, well beyond the scope of this manuscript.

\section{Summary}

Given the importance of software to modern science and the challenges in writing
high quality code, virtually all working scientists will need to ask for help
from time to time. The keys to doing so effectively are remarkably simple: ask
your questions succinctly and only after putting in some effort to solve the
problem yourself, be considerate of other people's time and effort, and look for
opportunities to contribute back to the community. Developers should make it
easy for users to behave this way by providing clear documentation (including
the appropriate venues to ask for help), encouraging participation, and
insisting on respectful behavior from users and developers alike.


\section{Checklists}
% Here is a single-column checklist that consists of multiple sub-checklists
\begin{Checklists}

\begin{checklist}{Good community member}
\begin{itemize}
\item Tries to solve problem themselves first
\item Asks for help in the right place
\item Writes informative bug reports
\item Cites and acknowledges software appropriately
\item Contributes to the community
\item Treats fellow members and developers with courtesy and respect
\end{itemize}
\end{checklist}

\begin{checklist}{Poor community member}
\begin{itemize}
\item Doesn't read the manual or search the internet before asking for help
\item Doesn't use the correct venue to ask for help
\item Writes vague or unhelpful bug reports, or doesn't respond to questions
\item Is rude or demanding when requesting support
\item Treats fellow community members disrespectfully
\end{itemize}
\end{checklist}

\begin{checklist}{A good community}
\begin{itemize}
\item Helps users solve their problems
\item Is friendly and supportive when responding to questions
\item Is receptive to suggestions and critiques, regardless of the source
\item Encourages participation from users of all experience levels
\item Encourages respectful treatment of all community members, and calls out bad behavior
\end{itemize}
\end{checklist}

\end{Checklists}








\section{Author Contributions}
%%%%%%%%%%%%%%%%
% This section mustt describe the actual contributions of
% author. Since this is an electronic-only journal, there is
% no length limit when you describe the authors' contributions,
% so we recommend describing what they actually did rather than
% simply categorizing them in a small number of
% predefined roles as might be done in other journals.
%
% See the policies ``Policies on Authorship'' section of https://livecoms.github.io
% for more information on deciding on authorship and author order.
%%%%%%%%%%%%%%%%
The initial version of this paper was written by Alan Grossfield.


% We suggest you preserve this comment:
For a more detailed description of author contributions,
see the GitHub issue tracking and changelog at \githubrepository.

\section{Other Contributions}
%%%%%%%%%%%%%%%
% You should include all people who have filed issues that were
% accepted into the paper, or that upon discussion altered what was in the paper.
% Multiple significant contributions might mean that the contributor
% should be moved to authorship at the discretion of the a
%
% See the policies ``Policies on Authorship'' section of https://livecoms.github.io for
% more information on deciding on authorship and author order.
%%%%%%%%%%%%%%%
Dr. Tod D. Romo reviewed the document for style and content. Dr. Justin A.
Lemkul made numerous helpful suggestions, particularly regarding language
accessibility. Dr. David H. Mathews provided several editorial suggestions. Dr.
Jeffrey Lewis had suggestions regarding searching for help.

The cover image was created by artistfymo (\url{https://linktr.ee/ArtistFrmYO}).

% We suggest you preserve this comment:
For a more detailed description of contributions from the community and others, see the GitHub issue tracking and Changelog at \githubrepository.

\section{Potentially Conflicting Interests}
%%%%%%%
%Declare any potentially competing interests, financial or otherwise
%%%%%%%

Alan Grossfield serves as a consultant to two companies: Moderna, Inc.
and Atelerix Life Sciences, and is a shareholder in Atelerix Life Sciences.

\section{Funding Information}
%%%%%%%
% Authors should acknowledge funding sources here. Reference specific grants.
%%%%%%%
This work supported in part by NIH R21GM138970 to Alan Grossfield.

\section*{Author Information}
\makeorcid

\bibliography{bibliography}

%%%%%%%%%%%%%%%%%%%%%%%%%%%%%%%%%%%%%%%%%%%%%%%%%%%%%%%%%%%%
%%% APPENDICES
%%%%%%%%%%%%%%%%%%%%%%%%%%%%%%%%%%%%%%%%%%%%%%%%%%%%%%%%%%%%

%\appendix


\end{document}
