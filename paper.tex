%%%%%%%%%%%%%%%%%%%%%%%%%%%%%%%%%%%%%%%%%%%%%%%%%%%%%%%%%%%%
%%% LIVECOMS ARTICLE TEMPLATE FOR TRAINING ARTICLE
%%% ADAPTED FROM ELIFE ARTICLE TEMPLATE (8/10/2017)
%%%%%%%%%%%%%%%%%%%%%%%%%%%%%%%%%%%%%%%%%%%%%%%%%%%%%%%%%%%%
%%% PREAMBLE
\documentclass[9pt,training]{livecoms}
% Use the 'onehalfspacing' option for 1.5 line spacing
% Use the 'doublespacing' option for 2.0 line spacing
% Use the 'lineno' option for adding line numbers.
% Use the "ASAPversion' option following article acceptance to add the DOI and relevant dates to the document footer.
% Use the 'pubversion' option for adding the citation and publication information to the document footer, when the LiveCoMS issue is finalized.
% The 'training' option for indicates that this is a training article.
% Omit the bestpractices option to remove the marking as a LiveCoMS paper.
% Please note that these options may affect formatting.

\usepackage{lipsum} % Required to insert dummy text
\usepackage[version=4]{mhchem}
\usepackage{siunitx}
\DeclareSIUnit\Molar{M}
\usepackage[italic]{mathastext}
\graphicspath{{figures/}}

%%%%%%%%%%%%%%%%%%%%%%%%%%%%%%%%%%%%%%%%%%%%%%%%%%%%%%%%%%%%
%%% IMPORTANT USER CONFIGURATION
%%%%%%%%%%%%%%%%%%%%%%%%%%%%%%%%%%%%%%%%%%%%%%%%%%%%%%%%%%%%

\newcommand{\versionnumber}{0.1}  % you should update the minor version number in preprints and major version number of submissions.
\newcommand{\githubrepository}{\url{https://github.com/GrossfieldLab/software_community}}

%%%%%%%%%%%%%%%%%%%%%%%%%%%%%%%%%%%%%%%%%%%%%%%%%%%%%%%%%%%%
%%% ARTICLE SETUP
%%%%%%%%%%%%%%%%%%%%%%%%%%%%%%%%%%%%%%%%%%%%%%%%%%%%%%%%%%%%
\title{How to be a good member of a scientific software community [Article v\versionnumber]}

\author[1*]{Alan Grossfield}
%\author[1,2\authfn{1}\authfn{3}]{Firstname Middlename Familyname}
%\author[2\authfn{1}\authfn{4}]{Firstname Initials Surname}
%\author[2*]{Firstname Surname}
\affil[1]{University of Rochester Medical Center, Department of Biochemistry and Biophysics}
%\affil[2]{Institution 2}

\corr{alan_grossfield@urmc.rochester.edu}{AG}  % Correspondence emails.  FMS and FS are the appropriate authors initials.
%\corr{email2@example.com}{FS}

\orcid{Alan Grossfield}{0000-0002-5877-2789}
%\orcid{Author 2 name}{EEEE-FFFF-GGGG-HHHH}

%\contrib[\authfn{1}]{These authors contributed equally to this work}
%\contrib[\authfn{2}]{These authors also contributed equally to this work}

%\presentadd[\authfn{3}]{Department, Institute, Country}
%\presentadd[\authfn{4}]{Department, Institute, Country}

\blurb{This LiveCoMS document is maintained online on GitHub at \githubrepository; to provide feedback, suggestions, or help improve it, please visit the GitHub repository and participate via the issue tracker.}

%%%%%%%%%%%%%%%%%%%%%%%%%%%%%%%%%%%%%%%%%%%%%%%%%%%%%%%%%%%%
%%% PUBLICATION INFORMATION
%%% Fill out these parameters when available
%%% These are used when the "pubversion" option is invoked
%%%%%%%%%%%%%%%%%%%%%%%%%%%%%%%%%%%%%%%%%%%%%%%%%%%%%%%%%%%%
\pubDOI{10.XXXX/YYYYYYY}
\pubvolume{<volume>}
\pubissue{<issue>}
\pubyear{<year>}
\articlenum{<number>}
\datereceived{Day Month Year}
\dateaccepted{Day Month Year}

%%%%%%%%%%%%%%%%%%%%%%%%%%%%%%%%%%%%%%%%%%%%%%%%%%%%%%%%%%%%
%%% ARTICLE START
%%%%%%%%%%%%%%%%%%%%%%%%%%%%%%%%%%%%%%%%%%%%%%%%%%%%%%%%%%%%

\begin{document}

\begin{frontmatter}
\maketitle

\begin{abstract}

Software is ubiquitous in modern science --- almost any project, in almost any
discipline, requires some code to work. However, many (or even most) scientists
are not programmers, and must rely on programs written and maintained by others.
As a result, a crucial but often neglected part of a scientist's training is
learning how to use new tools, and how to exist as part of a community of users.
This article will discuss key behaviors that can make the experience quicker,
more efficient, and more pleasant for the user and developer alike.


\end{abstract}

\end{frontmatter}




\section{Introduction}

Most practicing scientists (even the ones who write code as part of their
research) spend more time as consumers as opposed to producers of software.
Moreover, we are constantly learning new skills and solving new problems, which
often means learning to use new programs or new aspects of large packages. This,
combined with the complex nature of scientific software (and the often low
quality of the associated documentation) means we have to ask for help.
Sometimes it's because we don't know how to accomplish a specific task. Others,
it's because there's a feature we need that isn't implemented. Inevitably, there
are bugs.

In all of these cases, it is necessary to interact with the people who develop
and support the code. How you go about it has an enormous impact on your
likelihood of success and how you are viewed. Asking complete strangers for
help is often intimidating, especially the first time you do it. However, the
best way to do so is rarely taught explicitly --- at best, it's something one
picks up by example, from fellow lab members or the PI.  {\emph The goal of this
paper is to reveal the hidden curriculum -- what's expected of a software
consumer, how to ask for help, how to contribute productively to a software
community (regardless of whether you can write code).}

This article is written from a the perspective of a developer of academic
open-source scientific software, based on my personal experience as a developer
and user. I released my first piece of open source code, wham \cite{WHAM}, a
tool for analyzing umbrella sampling simulation,  during the summer of 2000.  My
group released LOOS\cite{Grossfield-2009, LOOS-JCC}, a suite of tools for
developing new tools to analyze molecular dynamics simulations, in 2008. Over
the years, I've interacted with hundreds of people who've tried to use one
package or the other. Most of these interactions have been positive, some
overwhelmingly so. Others were not. This article is an attempt to distill my
experiences into a concise set of recommendations.

Much of what I suggest is universal, but some points --- like the
reminder that the people providing the support are likely volunteers --- are
not.  Regardless, we view the interaction between developers and users as an
informal social contract, where each has obligations and expectations for the
other. Obviously, every software community is unique, and many have specific
conventions for interaction, but we hope that the recommendations we make are at
the very least a good starting point for new users of scientific software.



\section{Target Audience}

This paper is primarily aimed at junior scientists who are new to performing
research and inexperienced at asking for help. However, we hope it will be
valuable to anyone who uses software to do their research and struggles to ask
for help.

\section{Good practices in asking for help}

\subsection{Try to solve your own problem}

There's absolutely nothing wrong with needing help to figure out how to do
something with a new piece of software. Perhaps your use case wasn't considered
in the manual, or you're seeing behavior you don't expect. Perhaps you can't
even get it to install. Asking for help is very reasonable --- the developers
want people to use their software, and with that comes an implied obligation to
support those users. However, keep in mind that the developers of scientific
software are, for the most part, volunteering their time to provide that
support, and as such it's important to respect their time.

For this reason, asking the developers for help shouldn't be the \emph{first}
thing you do. Especially for software with a large community, chances are
someone else has run into similar difficulties, and you can probably save
yourself and the developers a lot of time if you look for solutions on your own
first. Indeed, quickly debugging your use of other people's software is one of
the major skills one develops when learning to do computational research.

The first step is to read the error message carefully, if there is one; this
obviously doesn't apply in all cases, but it's often applicable and far too
often skipped. It's true that many software packages have cryptic unhelpful
error messages, but far too often people write for help when the solution is
right in front of them. If the error message says ``Could not open file
foo.dat'', it's worth checking whether there is a file called ``foo.dat'' and if
not figuring out what was supposed to create it.

The second step is to check the manual. This might seem trivially obvious, but
you'd be astonished how many emails developers get that are directly answered in
the manual. If the manual is short, you can read the whole thing. If it's
longer, search for relevant-seeming keywords, scan the sections that seem to
pertain to your error message. If there's an FAQ (frequently asked questions)
list, check that first.

Next, search the internet for the error message; this works very well for
software with thousands of users, where it might very well deserve to be the
first option. It is less effective with less common packages, but it's an easy
thing to check, so you'd be foolish to skip it.

Finally, you can consider diving into the code. This decision depends very much
on the complexity of the package, your skill level, and how urgently you need
the solution. More often than not, comprehending the whole software package will
be too time consuming, but there are tricks and shortcuts one can use to at
least figure out where things might be going wrong. Figuring out where the error
is occurring can sometimes help you figure out why it happened, so examine the
stack trace if there is one. Alternatively, you can search the code base for the
error message, to see what drove the error; it's not uncommon for the comments
in the code to be clearer and more indicative of what could go wrong than the
error message itself. To be clear: as an end user, \emph{you are not obligated
to check the code}, but it can sometimes be a good shortcut to help you either
solve your problem on your own or give the developers what they need to solve
it. Moreover, if you do solve the problem, it's a good idea to share your
solution with the developers; most will be grateful to you for your help.

\subsection{Ask for help in the right place}

\subsection{Write a good bug report}

\subsection{Contribute to the community}

\subsection{Treat people with respect}

This should go without saying, but if you're asking for help it behooves you to
do so respectfully. We're not saying that you need to be obsequious. Rather,
approach every interaction with respect: respect for people's time, respect for
their ideas, respect for their identity. If you're asking for help, do so with
the perspective that the developer is volunteering their time to help you, with
minimal payoff to them.  If you're reporting a bug, you're almost certainly
frustrated; try not to take that frustration out on the developers. If you're
offering a new feature, keep in mind that it may not align with the developers'
vision for where the software is going.

If you're interacting with fellow users in a support forum, don't mock
questioners who are less experienced than you --- a question may seem foolish to
you, but even an uninformed or ill-directed question is an opportunity for
teaching; one of our goals for this document is that it become a resource to
educate new users of expected norms without consuming developers' time and
mental bandwidth.

\section{Obligations of software developers}

This paper has focused on what software consumers should do when asking for
help.  However, this does not mean that the developers of scientific software
are without responsibilities. If we want users to behave in a certain way, it
behooves us to tell them that. Users approaching us in good faith deserve to be
treated fairly and courteously.

connect to open science: FAIR principles say developers should want code to be
reusable. Publishing the code isn't enough, you need to make it usable as well.
That means support.

Treating everyone with courtesy and respect, regardless of their identity, is
essential if you want the software tool to have a vibrant community surrounding
it. Ensuring people aren't excluded is the fair and just way to proceed ---
making sure that racist, sexist, anti-LGBTQ, etc language and behavior is
unwelcome and unacceptable is simply the moral thing to do (it's also pragmatic
--- can you really afford to lose talented developers and users because they
don't feel welcome?) This doesn't just mean use of epithets in messages (though
that's clearly unacceptable). Keep an eye on how things are named, on jokes
present in the code or messages, on whose opinions are listened to and who is
ignored.  Above all, if someone tells you a particular behavior or situation is
harmful, \emph{believe them}. Something that feels minor (or is just  a joke) to
you may hit others very differently, and part of respect is understanding and
accounting for that.

\section{Checklists}
% Here is a single-column checklist that consists of multiple sub-checklists
\begin{Checklists}

\begin{checklist}{Good community member}
\begin{itemize}
\item Tries to solve problem themselves first
\item Asks for help in the right place
\item Writes informative bug reports
\item Cites and acknowledges software appropriately
\item Contributes to the community
\item Treats fellow members and developers with courtesy and respect
\end{itemize}
\end{checklist}

\begin{checklist}{Poor community member}
\begin{itemize}
\item Doesn't read the manual or search the internet before asking for help
\item Doesn't use the correct venue to ask for help
\item Writes vague or unhelpful bug reports, or doesn't respond to questions
\item Is rude or demanding when requesting support
\item Treats fellow community members disrespectfully
\end{itemize}
\end{checklist}

\begin{checklist}{A good community}
\begin{itemize}
\item Helps users solve their problems
\item Is friendly and supportive when responding to questions
\item Is receptive to suggestions and critiques, regardless of the source
\item Encourages participation from users of all experience levels
\item Encourages respectful treatment of all community members, and calls out disrespectful behavior
\end{itemize}
\end{checklist}

\end{Checklists}








\section{Author Contributions}
%%%%%%%%%%%%%%%%
% This section mustt describe the actual contributions of
% author. Since this is an electronic-only journal, there is
% no length limit when you describe the authors' contributions,
% so we recommend describing what they actually did rather than
% simply categorizing them in a small number of
% predefined roles as might be done in other journals.
%
% See the policies ``Policies on Authorship'' section of https://livecoms.github.io
% for more information on deciding on authorship and author order.
%%%%%%%%%%%%%%%%
The initial version of this paper was written by Alan Grossfield.


% We suggest you preserve this comment:
For a more detailed description of author contributions,
see the GitHub issue tracking and changelog at \githubrepository.

\section{Other Contributions}
%%%%%%%%%%%%%%%
% You should include all people who have filed issues that were
% accepted into the paper, or that upon discussion altered what was in the paper.
% Multiple significant contributions might mean that the contributor
% should be moved to authorship at the discretion of the a
%
% See the policies ``Policies on Authorship'' section of https://livecoms.github.io for
% more information on deciding on authorship and author order.
%%%%%%%%%%%%%%%


% We suggest you preserve this comment:
For a more detailed description of contributions from the community and others, see the GitHub issue tracking and changelog at \githubrepository.

\section{Potentially Conflicting Interests}
%%%%%%%
%Declare any potentially competing interests, financial or otherwise
%%%%%%%

Alan Grossfield serves as a consultant to two companies, Moderna Therapeutics
and Atelerix Life Sciences.

\section{Funding Information}
%%%%%%%
% Authors should acknowledge funding sources here. Reference specific grants.
%%%%%%%
This work supported in part by NIH R21GM138970 to AG.

\section*{Author Information}
\makeorcid

\bibliography{bibliography}

%%%%%%%%%%%%%%%%%%%%%%%%%%%%%%%%%%%%%%%%%%%%%%%%%%%%%%%%%%%%
%%% APPENDICES
%%%%%%%%%%%%%%%%%%%%%%%%%%%%%%%%%%%%%%%%%%%%%%%%%%%%%%%%%%%%

%\appendix


\end{document}
