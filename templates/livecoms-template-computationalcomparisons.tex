%%%%%%%%%%%%%%%%%%%%%%%%%%%%%%%%%%%%%%%%%%%%%%%%%%%%%%%%%%%%
%%% LIVECOMS ARTICLE TEMPLATE FOR BEST PRACTICES GUIDE
%%% ADAPTED FROM ELIFE ARTICLE TEMPLATE (8/10/2017)
%%%%%%%%%%%%%%%%%%%%%%%%%%%%%%%%%%%%%%%%%%%%%%%%%%%%%%%%%%%%
%%% PREAMBLE
\documentclass[9pt,comparison]{livecoms}
% Use the 'onehalfspacing' option for 1.5 line spacing
% Use the 'doublespacing' option for 2.0 line spacing
% Use the 'lineno' option for adding line numbers.
% Use the 'pubversion' option for adding the citation and publication information to the document footer.
% The 'bestpractices' option for indicates that this is a best practices guide.
% Omit the bestpractices option to remove the marking as a LiveCoMS paper.
% Please note that these options may affect formatting.

\usepackage{lipsum} % Required to insert dummy text
\usepackage[version=4]{mhchem}
\usepackage{siunitx}
\DeclareSIUnit\Molar{M}
\usepackage[italic]{mathastext}
\graphicspath{{figures/}}

%%%%%%%%%%%%%%%%%%%%%%%%%%%%%%%%%%%%%%%%%%%%%%%%%%%%%%%%%%%%
%%% IMPORTANT USER CONFIGURATION
%%%%%%%%%%%%%%%%%%%%%%%%%%%%%%%%%%%%%%%%%%%%%%%%%%%%%%%%%%%%

\newcommand{\versionnumber}{1.3}  % you should update the minor version number in preprints and major version number of submissions.
\newcommand{\githubrepository}{\url{https://github.com/myaccount/homegithubrepository}}  %this should be the main github repository for this article

%%%%%%%%%%%%%%%%%%%%%%%%%%%%%%%%%%%%%%%%%%%%%%%%%%%%%%%%%%%%
%%% ARTICLE SETUP
%%%%%%%%%%%%%%%%%%%%%%%%%%%%%%%%%%%%%%%%%%%%%%%%%%%%%%%%%%%%
\title{This is the title [Article v\versionnumber]}

\author[1*]{Firstname Middlename Surname}
\author[1,2\authfn{1}\authfn{3}]{Firstname Middlename Familyname}
\author[2\authfn{1}\authfn{4}]{Firstname Initials Surname}
\author[2*]{Firstname Surname}
\affil[1]{Institution 1}
\affil[2]{Institution 2}

\corr{email1@example.com}{FMS}  % Correspondence emails.  FMS and FS are the appropriate authors initials.
\corr{email2@example.com}{FS}

\contrib[\authfn{1}]{These authors contributed equally to this work}
\contrib[\authfn{2}]{These authors also contributed equally to this work}

\presentadd[\authfn{3}]{Department, Institute, Country}
\presentadd[\authfn{4}]{Department, Institute, Country}

\blurb{This LiveCoMS document is maintained online on GitHub at \githubrepository; to provide feedback, suggestions, or help improve it, please visit the GitHub repository and participate via the issue tracker.}

%%%%%%%%%%%%%%%%%%%%%%%%%%%%%%%%%%%%%%%%%%%%%%%%%%%%%%%%%%%%
%%% PUBLICATION INFORMATION
%%% Fill out these parameters when available
%%% These are used when the "pubversion" option is invoked
%%%%%%%%%%%%%%%%%%%%%%%%%%%%%%%%%%%%%%%%%%%%%%%%%%%%%%%%%%%%
\pubDOI{10.XXXX/YYYYYYY}
\pubvolume{<volume>}
\pubyear{<year>}
\articlenum{<number>}
\datereceived{Month, Day, Year}
\dateaccepted{Month, Day, Year}

%%%%%%%%%%%%%%%%%%%%%%%%%%%%%%%%%%%%%%%%%%%%%%%%%%%%%%%%%%%%
%%% ARTICLE START
%%%%%%%%%%%%%%%%%%%%%%%%%%%%%%%%%%%%%%%%%%%%%%%%%%%%%%%%%%%%

\begin{document}

\begin{frontmatter}
\maketitle

\begin{abstract}
This particular document provides a skeleton illustrating key sections for a Computational Comparisons document.
Please see the sample \texttt{sample-document.tex} in \url{github.com/livecomsjournal/article_templates/templates} for additional information on and examples of using the LiveCoMS LaTeX class.
Here we also assume familiarity with LaTeX and knowledge of how to include figures, tables, etc.; if you want examples, see the sample just referenced.

In your work, in this particular slot, please provide an abstract of no more than 250 words.
Your abstract should explain the main contributions of your article, and should not contain any material that is not included in the main text.
Please note that your abstract, plus the authorship material following it, must not extend beyond the title page or modifications to the LaTeX class will likely be needed.
\end{abstract}

\end{frontmatter}




\section{Introduction}

Here you would explain what problem you are tackling and briefly motivate your work; why is a comparison needed here?
What do the tools you are comparing do?

In this particular template, we have removed most of the usage examples which occur in \texttt{sample-document.tex} to provide a minimal template you can modify but we retain an example on algorithms and pseudocode.

\section{Methodology}

Here (and in other sections if needed) you would describe how you compared the software in question, with subsections as needed.
Perhaps you may also need to comment here on why you did not include certain pieces of software or how these might be included in the future.

\subsection{Physical systems and properties}

What did you calculate, on what systems? Why did you make that choice?

\subsection{Methodology for comparison}
How did you compute what you're comparing? 
What software versions did you use?
How would the software be accessed?


\subsubsection{Method selection}
How did you select which protocols to use? How are you certain these are representative? Would a naive user get similar performance? 

\subsubsection{Supporting files}
Are there run scripts or files others would need to reproduce your work? Are these provided in your GitHub repo, or how would they be accessed? 
How exactly would someone reproduce/extend this work?

\subsubsection{Error analysis}
You may wish to devote particular attention to error analysis.


\section{Results}

You would give results.


\section{Discussion and conclusions}

You may perhaps wish to revisit the issue of what performance a naive user might expect, here.



\section{Author Contributions}
%%%%%%%%%%%%%%%%
% This section mustt describe the actual contributions of
% author. Since this is an electronic-only journal, there is
% no length limit when you describe the authors' contributions,
% so we recommend describing what they actually did rather than
% simply categorizing them in a small number of
% predefined roles as might be done in other journals.
%
% See the policies ``Policies on Authorship'' section of https://livecoms.github.io
% for more information on deciding on authorship and author order.
%%%%%%%%%%%%%%%%

(Explain the contributions of the different authors here)

% We suggest you preserve this comment:
For a more detailed description of author contributions,
see the GitHub issue tracking and changelog at \githubrepository.

\section{Other Contributions}
%%%%%%%%%%%%%%%
% You should include all people who have filed issues that were
% accepted into the paper, or that upon discussion altered what was in the paper.
% Multiple significant contributions might mean that the contributor
% should be moved to authorship at the discretion of the a
%
% See the policies ``Policies on Authorship'' section of https://livecoms.github.io for
% more information on deciding on authorship and author order.
%%%%%%%%%%%%%%%

(Explain the contributions of any non-author contributors here)
% We suggest you preserve this comment:
For a more detailed description of contributions from the community and others, see the GitHub issue tracking and changelog at \githubrepository.

\section{Potentially Conflicting Interests}
%%%%%%%
%Declare any potentially competing interests, financial or otherwise
%%%%%%%

Declare any potentially conflicting interests here, whether or not they pose an actual conflict in your view.

\section{Funding Information}
%%%%%%%
% Authors should acknowledge funding sources here. Reference specific grants.
%%%%%%%
FMS acknowledges the support of NSF grant CHE-1111111.

\bibliography{livecoms-sample}

%%%%%%%%%%%%%%%%%%%%%%%%%%%%%%%%%%%%%%%%%%%%%%%%%%%%%%%%%%%%
%%% APPENDICES
%%%%%%%%%%%%%%%%%%%%%%%%%%%%%%%%%%%%%%%%%%%%%%%%%%%%%%%%%%%%

%\appendix


\end{document}
