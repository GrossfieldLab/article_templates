%%%%%%%%%%%%%%%%%%%%%%%%%%%%%%%%%%%%%%%%%%%%%%%%%%%%%%%%%%%%
%%% LIVECOMS ARTICLE TEMPLATE FOR BEST PRACTICES GUIDE
%%% ADAPTED FROM ELIFE ARTICLE TEMPLATE (8/10/2017)
%%%%%%%%%%%%%%%%%%%%%%%%%%%%%%%%%%%%%%%%%%%%%%%%%%%%%%%%%%%%
%%% PREAMBLE
\documentclass[9pt,bestpractices]{livecoms}
% Use the 'onehalfspacing' option for 1.5 line spacing
% Use the 'doublespacing' option for 2.0 line spacing
% Use the 'lineno' option for adding line numbers.
% Use the 'pubversion' option for adding the citation and publication information to the document footer.
% The 'bestpractices' option for indicates that this is a best practices guide.
% Omit the bestpractices option to remove the marking as a LiveCoMS paper.
% Please note that these options may affect formatting.

\usepackage{lipsum} % Required to insert dummy text
\usepackage[version=4]{mhchem}
\usepackage{siunitx}
\DeclareSIUnit\Molar{M}
\usepackage[italic]{mathastext}
\graphicspath{{figures/}}

%%%%%%%%%%%%%%%%%%%%%%%%%%%%%%%%%%%%%%%%%%%%%%%%%%%%%%%%%%%%
%%% IMPORTANT USER CONFIGURATION
%%%%%%%%%%%%%%%%%%%%%%%%%%%%%%%%%%%%%%%%%%%%%%%%%%%%%%%%%%%%

\newcommand{\versionnumber}{1.3}  % you should update the minor version number in preprints and major version number of submissions.
\newcommand{\githubrepository}{\url{https://github.com/myaccount/homegithubrepository}}  %this should be the main github repository for this article

%%%%%%%%%%%%%%%%%%%%%%%%%%%%%%%%%%%%%%%%%%%%%%%%%%%%%%%%%%%%
%%% ARTICLE SETUP
%%%%%%%%%%%%%%%%%%%%%%%%%%%%%%%%%%%%%%%%%%%%%%%%%%%%%%%%%%%%
\title{This is the title [Article v\versionnumber]}

\author[1*]{Firstname Middlename Surname}
\author[1,2\authfn{1}\authfn{3}]{Firstname Middlename Familyname}
\author[2\authfn{1}\authfn{4}]{Firstname Initials Surname}
\author[2*]{Firstname Surname}
\affil[1]{Institution 1}
\affil[2]{Institution 2}

\corr{email1@example.com}{FMS}  % Correspondence emails.  FMS and FS are the appropriate authors initials.
\corr{email2@example.com}{FS}

\contrib[\authfn{1}]{These authors contributed equally to this work}
\contrib[\authfn{2}]{These authors also contributed equally to this work}

\presentadd[\authfn{3}]{Department, Institute, Country}
\presentadd[\authfn{4}]{Department, Institute, Country}

\blurb{This LiveCoMS document is maintained online on GitHub at \githubrepository; to provide feedback, suggestions, or help improve it, please visit the GitHub repository and participate via the issue tracker.}

%%%%%%%%%%%%%%%%%%%%%%%%%%%%%%%%%%%%%%%%%%%%%%%%%%%%%%%%%%%%
%%% PUBLICATION INFORMATION
%%% Fill out these parameters when available
%%% These are used when the "pubversion" option is invoked
%%%%%%%%%%%%%%%%%%%%%%%%%%%%%%%%%%%%%%%%%%%%%%%%%%%%%%%%%%%%
\pubDOI{10.XXXX/YYYYYYY}
\pubvolume{<volume>}
\pubyear{<year>}
\articlenum{<number>}
\datereceived{Day Month Year}
\dateaccepted{Day Month Year}

%%%%%%%%%%%%%%%%%%%%%%%%%%%%%%%%%%%%%%%%%%%%%%%%%%%%%%%%%%%%
%%% ARTICLE START
%%%%%%%%%%%%%%%%%%%%%%%%%%%%%%%%%%%%%%%%%%%%%%%%%%%%%%%%%%%%

\begin{document}

\begin{frontmatter}
\maketitle

\begin{abstract}
This particular document provides a skeleton illustrating key sections for a Best Practices document. Please see the sample \texttt{sample-document.tex} in \url{github.com/livecomsjournal/article_templates/templates} for additional information on and examples of using the LiveCoMS LaTeX class.
Here we also assume familiarity with LaTeX and knowledge of how to include figures, tables, etc.; if you want examples, see the sample just referenced.

In your work, in this particular slot, please provide an abstract of no more than 250 words.
Your abstract should explain the main contributions of your article, and should not contain any material that is not included in the main text.
Please note that your abstract, plus the authorship material following it, must not extend beyond the title page or modifications to the LaTeX class will likely be needed.
\end{abstract}

\end{frontmatter}




\section{Introduction}

Here you would explain what problem you are tackling and briefly motivate your work.

In this particular template, we have removed most of the usage examples which occur in \texttt{sample-document.tex} to provide a minimal template you can modify; however, we retain a couple of examples illustrating more unusual features of our templates/article class, such as the checklists, and information on algorithms and pseudocode.

\section{Prerequisites}

Here you would identify prerequisites/background knowledge that are assumed by your work and your checklist which you view as critical, ideally giving links to good sources on these topics.
Checklists are normally focused on errors made by users with training and experience in molecular simulations, so you can assume a basic familiarity with the fundamentals of molecular simulations.

\section{Checklist}
Here we use a full-page checklist with multiple sections, so it will appear on a separate page of the sample PDF.
Other checklist formats are possible, as shown in the sample \texttt{sample-document.tex} in \url{github.com/livecomsjournal/article_templates/templates}.

Your checklist should include a succinct list of steps that people should follow when carrying out the task in question.
This is provided to ensure certain basic standards are followed and common but critical major errors are avoided.
Note that a checklist is not intended to cover \emph{all} important steps, but rather focus on the most common reasons for failure or incorrect results, or issues which are particularly crucial.


% This provides a checklist which
% - spans a full page
% - consists of multiple sub-checklists
% - exists on a separate page
% This style of checklist will be especially helpful if you want to encourage readers to print and use your checklist in practice, as they
% can easily print it without also printing other material from your manuscript. However, other styles of checklist are also possible (below).
\begin{Checklists*}[p!]

\begin{checklist}{First list}
\textbf{You can easily make full width checklists that take a whole page}
\begin{itemize}
\item Items in your checklist can and should reference sections where the cited issues are discussed in detail, such as Section~\ref{sec:reference_this}
\item Also remember
\item And finally
\end{itemize}
\end{checklist}

\begin{checklist}{Second list}
\textbf{This is some further description.}
\begin{itemize}
\item First thing
\item Also remember
\item And finally
\end{itemize}
\end{checklist}

\begin{checklist}{Third list}
\textbf{This is some further description.}
\begin{itemize}
\item First thing
\item Also remember
\item And finally
\end{itemize}
\end{checklist}

\begin{checklist}{Fourth list}
\textbf{This is some further description.}
\begin{itemize}
\item First thing
\item Also remember
\item And finally
\end{itemize}
\end{checklist}

\end{Checklists*}





\section{Rationale}

Your Rationale section, or sections, can follow or precede your checklist (we expect that often, following the checklist will be preferable) and provide the necessary rationale for the checklist, and act as more complete \emph{best practices} description.
This should include 1) significant detail as to the possible alternative ways to accomplish a given task, 2) description of advantages and disadvantages of the various approaches, and 3) significant literature documentation about reasons for choices.

\subsection{Algorithms and Pseudocode}
\label{sec:reference_this}

The \texttt{algpseudocode} and \texttt{algorithms} packages is loaded by the document class. \ALG{euclid} was taken directly from the package documentation. (Please do not load \texttt{algorithm2e}; it's not compatible with \texttt{algpseudocde} nor \texttt{algorithms}!)

\begin{algorithm}
\caption{Euclid's algorithm}\label{alg:euclid}
\begin{algorithmic}%[1]  %% uncomment to enable line numbers
\Procedure{Euclid}{$a,b$}\Comment{The g.c.d. of a and b}
   \State $r\gets a\bmod b$
   \While{$r\not=0$}\Comment{We have the answer if r is 0}
      \State $a\gets b$
      \State $b\gets r$
      \State $r\gets a\bmod b$
   \EndWhile\label{euclidendwhile}
   \State \textbf{return} $b$\Comment{The gcd is b}
\EndProcedure
\end{algorithmic}
\end{algorithm}





\section{Author Contributions}
%%%%%%%%%%%%%%%%
% This section mustt describe the actual contributions of
% author. Since this is an electronic-only journal, there is
% no length limit when you describe the authors' contributions,
% so we recommend describing what they actually did rather than
% simply categorizing them in a small number of
% predefined roles as might be done in other journals.
%
% See the policies ``Policies on Authorship'' section of https://livecoms.github.io
% for more information on deciding on authorship and author order.
%%%%%%%%%%%%%%%%

(Explain the contributions of the different authors here)

% We suggest you preserve this comment:
For a more detailed description of author contributions,
see the GitHub issue tracking and changelog at \githubrepository.

\section{Other Contributions}
%%%%%%%%%%%%%%%
% You should include all people who have filed issues that were
% accepted into the paper, or that upon discussion altered what was in the paper.
% Multiple significant contributions might mean that the contributor
% should be moved to authorship at the discretion of the a
%
% See the policies ``Policies on Authorship'' section of https://livecoms.github.io for
% more information on deciding on authorship and author order.
%%%%%%%%%%%%%%%

(Explain the contributions of any non-author contributors here)
% We suggest you preserve this comment:
For a more detailed description of contributions from the community and others, see the GitHub issue tracking and changelog at \githubrepository.

\section{Potentially Conflicting Interests}
%%%%%%%
%Declare any potentially competing interests, financial or otherwise
%%%%%%%

Declare any potentially conflicting interests here, whether or not they pose an actual conflict in your view.

\section{Funding Information}
%%%%%%%
% Authors should acknowledge funding sources here. Reference specific grants.
%%%%%%%
FMS acknowledges the support of NSF grant CHE-1111111.

\bibliography{livecoms-sample}

%%%%%%%%%%%%%%%%%%%%%%%%%%%%%%%%%%%%%%%%%%%%%%%%%%%%%%%%%%%%
%%% APPENDICES
%%%%%%%%%%%%%%%%%%%%%%%%%%%%%%%%%%%%%%%%%%%%%%%%%%%%%%%%%%%%

%\appendix


\end{document}
