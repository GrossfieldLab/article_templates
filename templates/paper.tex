%%%%%%%%%%%%%%%%%%%%%%%%%%%%%%%%%%%%%%%%%%%%%%%%%%%%%%%%%%%%
%%% LIVECOMS ARTICLE TEMPLATE FOR TRAINING ARTICLE
%%% ADAPTED FROM ELIFE ARTICLE TEMPLATE (8/10/2017)
%%%%%%%%%%%%%%%%%%%%%%%%%%%%%%%%%%%%%%%%%%%%%%%%%%%%%%%%%%%%
%%% PREAMBLE
\documentclass[9pt,training]{livecoms}
% Use the 'onehalfspacing' option for 1.5 line spacing
% Use the 'doublespacing' option for 2.0 line spacing
% Use the 'lineno' option for adding line numbers.
% Use the "ASAPversion' option following article acceptance to add the DOI and relevant dates to the document footer.
% Use the 'pubversion' option for adding the citation and publication information to the document footer, when the LiveCoMS issue is finalized.
% The 'training' option for indicates that this is a training article.
% Omit the bestpractices option to remove the marking as a LiveCoMS paper.
% Please note that these options may affect formatting.

\usepackage{lipsum} % Required to insert dummy text
\usepackage[version=4]{mhchem}
\usepackage{siunitx}
\DeclareSIUnit\Molar{M}
\usepackage[italic]{mathastext}
\graphicspath{{figures/}}

%%%%%%%%%%%%%%%%%%%%%%%%%%%%%%%%%%%%%%%%%%%%%%%%%%%%%%%%%%%%
%%% IMPORTANT USER CONFIGURATION
%%%%%%%%%%%%%%%%%%%%%%%%%%%%%%%%%%%%%%%%%%%%%%%%%%%%%%%%%%%%

\newcommand{\versionnumber}{0.1}  % you should update the minor version number in preprints and major version number of submissions.
\newcommand{\githubrepository}{\url{https://github.com/GrossfieldLab/article_templates}}

%%%%%%%%%%%%%%%%%%%%%%%%%%%%%%%%%%%%%%%%%%%%%%%%%%%%%%%%%%%%
%%% ARTICLE SETUP
%%%%%%%%%%%%%%%%%%%%%%%%%%%%%%%%%%%%%%%%%%%%%%%%%%%%%%%%%%%%
\title{How to be a good member of a scientific software community [Article v\versionnumber]}

\author[1*]{Alan Grossfield}
%\author[1,2\authfn{1}\authfn{3}]{Firstname Middlename Familyname}
%\author[2\authfn{1}\authfn{4}]{Firstname Initials Surname}
%\author[2*]{Firstname Surname}
\affil[1]{University of Rochester Medical Center, Department of Biochemistry and Biophysics}
%\affil[2]{Institution 2}

\corr{alan_grossfield@urmc.rochester.edu}{AG}  % Correspondence emails.  FMS and FS are the appropriate authors initials.
%\corr{email2@example.com}{FS}

\orcid{Alan Grossfield}{0000-0002-5877-2789}
%\orcid{Author 2 name}{EEEE-FFFF-GGGG-HHHH}

%\contrib[\authfn{1}]{These authors contributed equally to this work}
%\contrib[\authfn{2}]{These authors also contributed equally to this work}

%\presentadd[\authfn{3}]{Department, Institute, Country}
%\presentadd[\authfn{4}]{Department, Institute, Country}

\blurb{This LiveCoMS document is maintained online on GitHub at \githubrepository; to provide feedback, suggestions, or help improve it, please visit the GitHub repository and participate via the issue tracker.}

%%%%%%%%%%%%%%%%%%%%%%%%%%%%%%%%%%%%%%%%%%%%%%%%%%%%%%%%%%%%
%%% PUBLICATION INFORMATION
%%% Fill out these parameters when available
%%% These are used when the "pubversion" option is invoked
%%%%%%%%%%%%%%%%%%%%%%%%%%%%%%%%%%%%%%%%%%%%%%%%%%%%%%%%%%%%
\pubDOI{10.XXXX/YYYYYYY}
\pubvolume{<volume>}
\pubissue{<issue>}
\pubyear{<year>}
\articlenum{<number>}
\datereceived{Day Month Year}
\dateaccepted{Day Month Year}

%%%%%%%%%%%%%%%%%%%%%%%%%%%%%%%%%%%%%%%%%%%%%%%%%%%%%%%%%%%%
%%% ARTICLE START
%%%%%%%%%%%%%%%%%%%%%%%%%%%%%%%%%%%%%%%%%%%%%%%%%%%%%%%%%%%%

\begin{document}

\begin{frontmatter}
\maketitle

\begin{abstract}

Software is ubiquitous in modern science --- almost any project, in almost any
discipline, requires some code to work. However, many (or even most) scientists
are not programmers, and must rely on programs written and maintained by others.
As a result, a crucial but often neglected part of a scientist's training is
learning how to use new tools, and how to exist as part of a community of users.
This article will discuss key behaviors that can make the experience quicker,
more efficient, and more pleasant for the user and developer alike.


\end{abstract}

\end{frontmatter}




\section{Introduction}

Most practicing scientists (even the ones who write code as part of their
research) spend more time as consumers as opposed to producers of software.
Moreover, we are constantly learning new skills and solving new problems, which
often means learning to use new programs or new aspects of large packages. This,
combined with the complex nature of scientific software (and the often low
quality of the associated documentation) means we have to ask for help.
Sometimes it's because we don't know how to accomplish a specific task. Others,
it's because there's a feature we need that isn't implemented. Inevitably, there
are bugs.

In all of these cases, it is necessary to interact with the folks who develop
and support the code. How you go about it has an enormous impact on your
likelihood of success and how you are viewed, but this is not something that is
explicitly taught in most places.  {\emph The goal of this work is to reveal the hidden
curriculum -- what's expected of a software consumer, how to ask for help, how
to contribute productively to a software community (regardless of whether you
can write code).}

There lots of articles on best practices in software development, including open
source.

\section{Prerequisites}

Training articles should clearly state the target audience and knowledge prerequisites.
Key prerequisites should be noted in the article abstract to permit readers to rapidly ascertain an articles suitability.

\subsection{Background knowledge}

Although the authors may imagine a particular career-level (e.g., undergraduate or graduate), given the diversity of disciplinary curricula, it is more important to specify precisely any knowledge prerequisites (e.g., vector calculus, basic thermodynamics).

\subsection{Software/system requirements}

If a particular software or programming environment plays a central role in the article, that should be specified.

\section{Content and links}

A training article may on additional files and materials; clearly indicate where and how these are available, with links, and how they are being archived for the long-term and maintained so they stay current.
You will likely want to reference your GitHub repository as a central point to access all of this information, and then the GitHub repository may link out to other content as needed.

\section{Checklists}
% Here is a single-column checklist that consists of multiple sub-checklists
\begin{Checklists}

\begin{checklist}{Good community member}
\begin{itemize}
\item Tries to solve problem themselves first
\item Asks for help in the right place
\item Writes informative bug reports
\item Cites and acknowledges software appropriately
\item Contributes to the community
\item Treats fellow members and developers with courtesy and respect
\end{itemize}
\end{checklist}

\begin{checklist}{Poor community member}
\begin{itemize}
\item Doesn't read the manual or search the internet before asking for help
\item Doesn't use the correct venue to ask for help
\item Writes vague or unhelpful bug reports, or doesn't respond to questions
\item Is rude or demanding when requesting support
\item Treats fellow community members disrespectfully
\end{itemize}
\end{checklist}

\begin{checklist}{A good community}
\begin{itemize}
\item Helps users solve their problems
\item Is friendly and supportive when responding to questions
\item Is receptive to suggestions and critiques, regardless of the source
\item Encourages participation from users of all experience levels
\item Encourages respectful treatment of all community members, and calls out disrespectful behavior
\end{itemize}
\end{checklist}

\end{Checklists}








\section{Author Contributions}
%%%%%%%%%%%%%%%%
% This section mustt describe the actual contributions of
% author. Since this is an electronic-only journal, there is
% no length limit when you describe the authors' contributions,
% so we recommend describing what they actually did rather than
% simply categorizing them in a small number of
% predefined roles as might be done in other journals.
%
% See the policies ``Policies on Authorship'' section of https://livecoms.github.io
% for more information on deciding on authorship and author order.
%%%%%%%%%%%%%%%%
The initial version of this paper was written by Alan Grossfield.


% We suggest you preserve this comment:
For a more detailed description of author contributions,
see the GitHub issue tracking and changelog at \githubrepository.

\section{Other Contributions}
%%%%%%%%%%%%%%%
% You should include all people who have filed issues that were
% accepted into the paper, or that upon discussion altered what was in the paper.
% Multiple significant contributions might mean that the contributor
% should be moved to authorship at the discretion of the a
%
% See the policies ``Policies on Authorship'' section of https://livecoms.github.io for
% more information on deciding on authorship and author order.
%%%%%%%%%%%%%%%


% We suggest you preserve this comment:
For a more detailed description of contributions from the community and others, see the GitHub issue tracking and changelog at \githubrepository.

\section{Potentially Conflicting Interests}
%%%%%%%
%Declare any potentially competing interests, financial or otherwise
%%%%%%%

Alan Grossfield serves as a consultant to two companies, Moderna Therapeutics
and Atelerix Life Sciences.

\section{Funding Information}
%%%%%%%
% Authors should acknowledge funding sources here. Reference specific grants.
%%%%%%%
This work supported in part by NIH R21GM138970 to AG.

\section*{Author Information}
\makeorcid

\bibliography{bibliography}

%%%%%%%%%%%%%%%%%%%%%%%%%%%%%%%%%%%%%%%%%%%%%%%%%%%%%%%%%%%%
%%% APPENDICES
%%%%%%%%%%%%%%%%%%%%%%%%%%%%%%%%%%%%%%%%%%%%%%%%%%%%%%%%%%%%

%\appendix


\end{document}
